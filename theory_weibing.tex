\section{治未病理论概述}
治未病是几千年来中国人在集体生活实践中发展而来的一种保持身体调达、心理安康的理论和实践方法。它基于一种忧患意识,表达了人们对于健康生活的渴望。随着时代的进步,治未病理论体系建立、发展、丰富,围绕理论逐步形成了一种有效的实践体系。治未病理论在总的原则上符合“三因制宜”,分别为顺应节气、符合地域、适应个人等三个层次,在具体的操作方法上有饮食调理、运动调理、精神调理等多种手段。

\subsection{治未病理论的历史溯源}
关于治未病的最早记载出现在《黄帝内经》之中,主要提出了未病先防和既病防变两个方面,为治未病理论提供了最初的理论框架。《素问·四气调神大论》曰:“圣人不治已病治未病,不治已乱治未乱,此之谓也。夫病已成而后药之,乱已成而后治之,譬犹渴而穿井,斗而铸锥,不亦晚乎!”\cite{南京中医学院医经教研组1981黄帝内经素问译释},说明养生的关键在于未病先防。《素问·阴阳应象大论》云 :“故邪风之至,疾如风雨 ,故善治者治皮毛,其次治肌肤,其次治筋脉, 其次治六腑, 其次治五脏 。治五脏者,半死半生也。”\cite{南京中医学院医经教研组1981黄帝内经素问译释},说明早期诊治可以得到更好的疗效,如果病邪已深,则治疗的预后不良。

后世的医家对《内经》提出的这一体系不断丰富,在历代医家的书籍中皆有论述。比如汉代的张仲景提出了六经辨证的体系,说明了疾病在脏腑之间的传变关系,提倡早期治疗,防微杜渐。唐代孙思邈则提出了“上医医未病之病,中医医欲病之病,下医医已病之病”\cite{李俊德2008中医必读百部名著}的说法,这一说法如今广为人知,可见防病、防变的思想影响深远。元代医家朱震亨在《格致余论》中提出:“与其求疗于有病之后,不若摄养于无疾之先;盖疾成而后药者,徒劳而已,……,未病而先治,所以明摄生之理。"\cite{朱震亨2008格致余论}清代名医叶天士在温病论治方面发展出卫气营血理论,揭示了外感温热病由表入里、由浅入深的一般规律,是对既病防变的又一重要补充。另外叶天士本人对于”未病先防“极为重视,他在《温热论》中指出:“务在先安未受邪之地。”\cite{叶桂2007温热论}

到了近现代,治未病理论体系得到了新的补充,主要体现在”欲病先防“和”病后防复“两个方面。现代人由于生活节奏紧张、精神压力大、饮食起居无节等因素,往往处于一种亚健康状态。这种状态往往持续时间长,无明显的疾病表观,但是此时正气不充,容易为外邪牵动内因而发病,处于”即将生病”的趋势之中,因而对于此类易感人群应该和健康的人群有所区分,由原先的“未病先防”转为“欲病先防”。而病后防复分为两个方面,其一是宋为民在其《未病论》\cite{宋为民1992未病论}中指出的“潜病未病态”,也就是对于季节性发作的疾病,例如慢性支气管炎在未发作时人的状态。而对于这样的疾病,“冬病夏治”是很好的办法,以防来年病情的反复或加重。其二是对于大病初愈之人,特别是恢复速度不快的儿童、老人等群体,应当嘱托其饮食起居上的注意,以防“食复”、“劳复”、“再受外邪而并发”等等。

\subsection{治未病理论的三个层次}
治未病理论在根本上离不开中医学的特色,一是以阴阳为根本进行调节以恢复人体的平衡,所谓“阴平阳秘,精神乃治”;二是天人合一下的整体观念,《灵枢·岁露》曰:“人与天地相参也,与日月相应也。“,揭示出人的生命活动与自然息息相关。在这样的大原则的指导下,治未病理论发展出和周围的环境相互协调的三个具体原则。
\begin{enumerate}
\item 顺应节气:《素问·四气调神大论》曰:“夫四时阴阳者,万物之根本也,所以圣人春夏养阳,秋冬养阴,以从其根。”\cite{南京中医学院医经教研组1981黄帝内经素问译释}古时以农耕文化为主体,人以四时节律为行为准则,以一天为周期看则“日出而作,日落而归”,以一年为周期看则“春种秋收”,具体农事活动又有二十四节气的指导。这样的顺应节气的思想在治未病理论当中也有相关的体现。以春季为例,《素问·四气调神大论》曰“春三月,此谓发陈,天地俱生,万物以荣,夜卧早起,广步于庭,被发缓形,以使志生,生而勿杀,予而勿夺,赏而勿罚,此春气之应,养生之道也。逆之则伤肝,夏为寒变,奉长者少。”\cite{南京中医学院医经教研组1981黄帝内经素问译释},可见自然节律和人养生防病是密切相关的,人应因时而动。

\item 符合地域:俗话说一方水土养一方人,我国幅员辽阔,地形复杂,生态环境多样。地理环境的差异造就了不同地域之中人体质、生活方式等等方面的差异。《灵枢·阴阳二十五人》根据人体各方面的特征进行系统分类,认为方位和五行相符,例如东方之人似木形人,尽管如今看来不够科学,但是表示古人已经认识到了不同地域之间人体的差异。《素问·五常政大论》指出“一州之气,生化寿夭不同……,高下之理,地势使然也。”\cite{南京中医学院医经教研组1981黄帝内经素问译释}也说明了水土不同对于人的寿命的影响。治未病也要考虑所在地域的特点,这一点也在实践中得到证实。例如四川地区喜爱食用折耳根,能够清除湿热,和四川地势较低,气候湿热的特征相符合,是饮食防病的良好实践。

\item 适应个人:人生来禀赋不同,体质有殊。王琦教授在既往体质分类研究的基础上 ,进一步完善了体质分类系统 ,将人体体质分为平和质、阴虚质、阳虚质、
阳盛质、气虚质、瘀血质、痰湿质、湿热质、气郁质9种基本中医体质类型\cite{王琦20059},为治未病提供了更加个性化的分类基础。治未病在实际操作中也需要依从个人实际情况制定相应的合理方案,不可一概而论以免伤身,违背了保全身体的初衷。
\end{enumerate}
\subsection{治未病理论的实践手段}
\begin{itemize}
\item 饮食调理:饮食调理指运用治未病理论中的饮食养生理论构建合理的膳食结构。人以五谷为养,合理的日常饮食和人的身体健康密切相关。治未病《素问·藏气法时论》提出“五谷为养,五果为助,五畜为益,五菜为充”。五谷、五果、五畜、五菜泛指各种谷物瓜果、肉类蔬菜,核心即为膳食平衡。这一理念在当前菜品逐渐丰富的情况下显得尤为重要:“高粱之变,足生大丁”,现代人喜欢流连于肥甘厚味的食肆,呼朋引伴推杯换盏之间人的健康就逐渐受损,长此以往五脏失调,疾病百生。

因此不论是对于身体健康的人,还是已经受损,处于亚健康状态或病后状态的人群而言,治未病理论能提供合理的理论支持,帮助人们回复膳食平衡。主要是对饮食种类的选择有所偏好,遵循五味调和的原则选取食物,并且适应一个人所处的年龄、体质、地域等等。
\item 运动调理:运动调理指借助特定的锻炼方法来达到活动筋骨、调畅气血,保持生命平衡的作用。《格致余论》云:“天之为物,故恒于动,人之有生亦恒于动”。治未病理论重视运动对于防治疾病的重要性,也为此发明出很多运动调理的方法。最早《内经》就提出练习气功来保全身体、防治疾病。东汉名医华佗依据导引术发明“五禽戏”,模仿动物姿态,效法自然来达到养生防病的目的。同样,一直为中老年群体喜爱的简化太极也因其动作和缓,张弛有度,能够调理呼吸、流通气血得到了群众的肯定。对于病后的人群,则遵医嘱进行适当的运动,也利于病后恢复。

\item 精神调理:精神调理指在治未病理论的指导下疏导情绪,保持心情调达和舒畅。
世卫组织已经提出健康的概念不仅指身体的健康,还有情志状态健康以及适应社会。中医极其重视人的精神情志的平衡状态,《内经·上古天真论》提出,“恬淡虚无,真气从之,精神内守,病安从来。”在当前竞争性的社会背景下,人们的生活受到来自学习、工作、人际交往等多个来源的压力。如果不能妥当处理好自身的情绪,就容易陷入“亚健康”状态,受到精神不振,情绪低落,反应迟钝和慢性病等等困扰。

因而掌握一些常见的疏导情绪的食材、药材配伍对于人的精神健康非常重要,特别是肝气理论擅于调节人的情志,辅以人的自主调节,在新时代有广泛的应用前景。
\end{itemize}

\section{当前社区健康体系建设}
\subsection{社区和社区卫生服务的定义}
\subsubsection{社区}社会学家普遍认为一个社区应该包括一定数量的人口、一定范围的地域、一定规模的设施、一定特征的文化、一定类型的组织。换言之,社区是宏观社会的缩影,是聚居在一定地域范围内的人们所组成社会生活共同体。
\subsubsection{社区卫生服务}
社区卫生服务是社区服务中的一种最基本、普遍的服务,是由全科医生为主要卫生人力的卫生组织或机构所从事的一种社区定向的卫生服务,与医院定向的专科服务有所不同,它是社区建设的重要组成部分,以人的健康为中心、以家庭为单位、社区为范围、需求为导向,以妇女、儿童、老年人、慢性病患者、残疾人、低收入居民为重点,以解决社区主要卫生为题,满足基本医疗卫生服务需求为目的,融预防、医疗、保健、康复、健康教育和计划生育技术服务等为一体的,有效的、经济的、方便的、综合的、连续的基层卫生服务。 
\subsection{发展与现状}
从1985年的医疗卫生体制改革以来,“看病难、看病贵”的问题一直困扰着我国的医疗卫生体制。与西方成熟的社区医疗卫生模式相比,我国社区医疗起步较晚,于1996年开始试点,全面推动社区卫生服务体系的建设则在2000年开始进行。我国城市社区卫生服务的多年实践经验与国外社区卫生服务发展的成功经验告诉我们,发展社区卫生服务可以有效地解决居民“看病难、看病贵”的问题。一方面,发展社区卫生服务可以分流大医院的患者,解决居民“看病难”的问题,另一方面,社区卫生服务机构的预防保健、健康教育等服务可以有效提高居民自我保健意识,降低疾病发病率,同时,社区卫生服务机构提供的基本医疗服务价格比大医院低的多,可以在一定程度上降低居民的医疗支出,解决“看病贵”的问题。
\subsubsection{卫生资源分布不合理}
主要表现为卫生资源的分配比例和利用率的结构失衡。根据世界卫生组织的统计,社区卫生服务机构理应获得大部分的卫生资源。但是我国的情况正好相反:长期以来,在我国城市卫生服务工作中一直存在着重视综合型医疗卫生机构建设、轻视基层医疗卫生机构建设的情况。
\subsubsection{社区卫生服务补偿机制落实不足}
社区卫生服务的主要职责是向社区居民提供公共卫生服务和基本医疗服务,其本质是一项公益性的事业。社区卫生服务项目的开展需要政府在财政上给予充分的支持和保证。但目前我国政府对社区卫生服务机构的财政投入明显不足,社区医疗保险机构也没有将全部的社区卫生服务机构纳入医保定点机构,社区卫生服务机构的资金来源紧张,导致其难以有效地开展相关社区卫生服务项目。
\subsubsection{人力资源约束成社区卫生服务事业发展的瓶颈}
目前社区卫生服务机构中执业医生数量仅为综合性医院的4.5\%,且其中很多并不具有全科医生职业资格。公共卫生人员少,公共卫生的综合性技能差,预防和保健工作成为了社区卫生服务的弱项。因此,人才成为制约社区卫生服务发展的“瓶颈”。同时,全科医生培养模式不能适应社区卫生服务的发展要求,社区卫生服务机构的人力资源制度落后。
\subsubsection{缺少有效的监督机制}
鉴于我国医疗信息资源在医患之间严重不对称的现状,必须有一个监督主体的存在,对社区卫生服务机构的相关工作进行有效的监督。
\section{治未病理论在治未病社区健康建设的应用和前景}

\section{治未病“知信行”调查问卷的设计}
该问卷主要评价受调查者对中医治未病理论对于中医治未病理论的客观认知、所持信念以及行动意愿,以得出描述性的分析。设计的条目先进行征集,经编写后交付相关专家进行评议,并进行问卷重测以检验其信度。先前的研究有陈建伟对社区老年人进行的调查\cite{cjw_1_2009},其中拟定了一份“知信行”问卷可以进行部分参考。
