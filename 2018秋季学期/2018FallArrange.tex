\documentclass{article}
\usepackage{ctex}
\usepackage{url}
\usepackage{setspace}
\usepackage[left=2.5cm,right=2.5cm]{geometry}
\title{2018挑战杯工作计划以及相关安排}
\author{付裕刚}
\date{\today}
\begin{document}
    \maketitle
    \zihao{-4}
    \section{18.11.17更新 | 工作安排}
    首先,用一个列表总结一下依照先前设计的问卷计划安排写作的内容,其中带有星号的为已写。
    \begin{itemize}
        \item 治未病含义介绍
        \item 治未病的实践手段
        \item 整体观*
        \item 天人合一
        \item 五运六气(任应秋)
        \item 调畅情志和相关人群
        \item 食物寒凉说
        \item 刮痧介绍
        \item 推拿介绍
        \item 拔火罐介绍
        \item 中西预防医学对比
        \item 中医科学化之争
        \item 运动养生*保健操、导引术
        \item 饮食养生*大闸蟹、养生茶、冬枣
        \item 穴位养生*耳穴按摩
        \item 精神养生*脱发
        \item 节气养生*霜降
        \item 中医基础系列*前言、藏象、气机、五行
    \end{itemize}

其次,我们安排一下下一周大家要写的内容,黄晨写一下刮痧介绍,周宇写推拿介绍、徐天泓写拔火罐介绍,周元淳写峨眉伸展功的延续,我来写治未病含义和气机导言。

另外提及一下写作流程:我们做推送很多材料都是对于书中、网络上的材料进行重新编排和再次编辑,因此在写作的一开始需要有一写作的大纲统领。这样一是可以把握篇幅,二是可以把握写作的节奏,三是方便后续的修订。然后再进行各个部分的写作,这时候要把握所用材料的核心意思,并且把他的话语篇幅拉长,句式缩短,以适应微信文章的特点。同样,这一步目的在于降低语言理解的难度,摘掉原来的术语帽子。最后对于全文的逻辑顺序进行调整,编辑排版、推送收工。一般一周可以产出2篇推文,如果涉及的内容自己也很不熟悉,速度就会大打折扣;在考试周可以有一篇。总结一下:
\begin{enumerate}
    \item 编写大纲,收集材料
    \item 理解材料,重编材料
    \item 去术语化,突出重点
    \item 简明排版,平台推送
\end{enumerate}
由于大家实在是懒得学习Github的上传,因此最低要求就是能够把我们的库下载下来,库的地址是\url{https://github.com/BatheFu/backupclcp/}目前的文件在“秋季学期”文件夹内都能找到,包括综述pdf和之前答辩的PPT。此外,每周把写好的用txt格式发我邮箱即可,方便进行存档。
    \section*{18.10.5更新 | 工作安排}
    \begin{enumerate}
        \item 10.14之前需要交的申报书、PPT(3分钟演示)
        \item 统计报告修改
        \item 微信文章写作
    \end{enumerate}

    已经完成的计划文章篇目:
    
    序章;整体观;阴阳1、2;五行1、2;气血1;藏象1
    
    未完成的文章篇目:
    
    藏象2、3;气机升降;津液;风;火;暑;湿;燥;寒
    
    养生系列完成情况:共6, 徐天泓(5)、周宇(1)
    \subsection{注意点}
    \begin{itemize}
        \item 使用邮箱或者Git提交,文件格式为txt,每周提交两篇,尽量在工作日完成。
        \item 目前通过OCR识别等手段多屯一屯养生的稿子,写作难度相对小。排版依照群里的要求。
        \item 中医系列的写作,还存在以下的几个问题,一是部分文章中基语言的痕迹还是有不少,
    \end{itemize}
    \section{整体计划简述}
    我们的项目名称暂定为“治未病知信行调查以其互联网+路径探索”。如题,关键词一个是知信行,一个就是互联网+。
    
    理论层面,如同主文件(main.pdf)所写的,我们已经写了的是治未病的理论概述,知信行理论介绍,社区健康体系(已经不再需要),以及问卷设计。还需要补充的是治未病理论部分,增添和预防医学的区别,增添治未病理论和实际服务的桥梁:具体落实的项目,以及治未病的需求分析。另外还需要补充微信健康类平台特点、文章特点、传播策略特点等等,同样是做文献的梳理和综述。
    
    实际操作层面,需要完成的任务有两个,一是对于知信行现状进行问卷调查,并且得出一份统计描述报告,二是对于粉丝进行知信行网络干预,并且在干预的三个月后进行效果评估。
    
    需要解释的问题是,为什么我们要费这么大劲做平台,写文章,然后向其他平台,例如一临另一个大创团队“知国医”进行投稿呢?这是和我们研究希望涉及的人群有关的。我们的预期读者是对于治未病(狭义上讲即预防保健)有兴趣的普罗大众,次一级的读者是对于中医知识有一定兴趣的大众。他们的相同点是都属于中青年群体,有治未病产品或者服务的一定消费能力和欲望,并且能够使用微信,定期查看某些公众号推送内容。但是读者群体也会有一定差异,其中主要预期读者的文化程度较次一级预期读者水平较低一些,他们的核心关注点在于“养生内容”上。次一级预期读者是主要预期读者的子集,他们不仅关注养生,还关注其他的中医内容。因此创作的主要内容是养生热点,但是每篇文章同样提供外部链接以介绍其他中医知识。
    \section{统一编写方式}
    \begin{spacing}{0.7}
    本项目可能涉及的软件
    \begin{itemize}
        \item *Git/Github
        \item *Notepad++
        \item TexStudio
        \item Endnote/NoteExpress
    \end{itemize}
**其中星号为必须品
\newline
\end{spacing}
本项目涉及不少文字编写工作,需要统一的协作平台来提高效率,统一版本。为此使用Git开源版本控制工具以及Github线上平台完成协作任务。对于文字稿和\LaTeX 论文稿的编写统一使用Notepad++软件,以避免Microsoft记事本存在的Unicode编码干扰问题。

具体方法见统一库\url{https://github.com/BatheFu/backupclcp}当中名为,统一提交标准的Word文档。
    
    另外,对于负责文献梳理和综述写作的同学,可以使用Endnote或者NoteExpress文献管理工具来梳理已经阅读过的文献,并且管理做过的标注。对于读过的文献,修改统一库中的refer.bib文件并且上传提交。其中Bibtex的格式可以在百度学术获得,也可以在上述提到的《统一提交标准》当中看到图文说明。
    \section{人员安排}
    文献梳理和综述书写:付裕刚
    
    微信推文书写:徐天泓、周元淳、周宇、付裕刚
    
    约稿参与:朱贺
    
    统计报告:付裕刚
    
    \section{九月计划安排}
    \begin{itemize}
        \item 完成文献综述部分的补充,建立起完整的基本架构,梳理研究的脉络。
        \item 使用搜狗微信搜索平台跟踪近三个月的养生话题热点,并且根据此写主推送稿20篇。
        \item 根据教材、名家论文集、医案、医话等完成基本概念的“词条科普式”文章,作为治未病养生推送的增补材料40篇。
        \item 每周工作时间4-5小时,利用好水课/课间时间会变得轻松。
    \end{itemize}

谢谢各位对我的理解和支持!


\end{document}