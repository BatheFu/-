\documentclass{article}
\usepackage{ctex}
\usepackage{graphicx}
\usepackage[table,xcdraw]{xcolor}
\usepackage[left=3cm,right=3cm]{geometry}
\graphicspath{{images/}}
\usepackage[backend=biber,style=gb7714-2015]{biblatex}
\addbibresource[location=local]{refer.bib}	
\title{南京市社区治未病“知信行”调查统计报告}
\author{付裕刚}
\date{\today}
\begin{document}
    \maketitle
\section{对象和方法}
\subsection{对象}
选取南京市五个社区,分属不同的五个区划,每个社区的人数在6-8千人。统计各社区楼栋数,根据居委会的提示,排除青年人或老年人占绝大多数的楼层,随机选取居民进行自填式问卷调查。共调查192户,回收问卷265份。其中男性85人,女性180人,比例为32:68。
\subsection{方法}
\subsubsection{问卷设计}
根据先前进行的访谈,以及可进行参考的文献\cite{cjw_1_2009}\cite{wcy2011}\cite{kam2005}并咨询专家自行设计问卷,包括治未病知识、信念和行为三个维度。

其中,知识维度包括10条单选和2条多选。其中第1题为筛选题,不计分。
第2题承接第1题的引导,为多选,考察治未病的含义,每个选项计1分,共三分。
第3-10题采用4级评分法(正确=2分,部分正确=1分,错误=0分,不清楚=NA)。共计19分。
第11题为多选题,了解了知识获取的途径,不计分。故知识维度共19分。

信念维度共8条单选,序号为12-19,使用4级评分法(相信/愿意/喜欢=2分,部分相信/部分愿意/部分喜欢=1分,不相信/不愿意/不喜欢=-2分、不好说=NA)。共计16分。

行为维度共6题。序号为20-26,其中第20题设置跳题逻辑,选择(A=0次)跳过21题,分值为0、1、2、3。21为多选,考察治疗手段。22-26为单选,同样采用差额计分,最高项2分。其中26有4小题。共计24分。

具体可见下表
\newpage
   
         \begin{minipage}[b]{0.5\linewidth}
    \centering
    \begin{tabular}{cccl}
        \cellcolor[HTML]{FFFFFF}题号   & \cellcolor[HTML]{FFFFFF}维度 & \cellcolor[HTML]{FFFFFF}题目类型 & 是否有跳题 \\
        \cellcolor[HTML]{FFFFFF}1    & \cellcolor[HTML]{FFFFFF}知识 & \cellcolor[HTML]{FFFFFF}单选   & 是     \\
        \cellcolor[HTML]{FFFFFF}2    & \cellcolor[HTML]{FFFFFF}知识 & \cellcolor[HTML]{FFFFFF}多选   & 否     \\
        \cellcolor[HTML]{FFFFFF}3-10 & \cellcolor[HTML]{FFFFFF}知识 & \cellcolor[HTML]{FFFFFF}单选   & 否     \\
        \cellcolor[HTML]{FFFFFF}11   & \cellcolor[HTML]{FFFFFF}知识 & \cellcolor[HTML]{FFFFFF}多选   & 否     \\
        \multicolumn{1}{l}{12-19}    & \multicolumn{1}{l}{信念}     & \multicolumn{1}{l}{单选}       & 否     \\
        \multicolumn{1}{l}{20}       & \multicolumn{1}{l}{行为}     & \multicolumn{1}{l}{单选}       & 是     \\
        \multicolumn{1}{l}{21}       & \multicolumn{1}{l}{行为}     & \multicolumn{1}{l}{多选}       & 否     \\
        \multicolumn{1}{l}{22-26}    & \multicolumn{1}{l}{行为}     & \multicolumn{1}{l}{单选}       & 否    
    \end{tabular}

\end{minipage}
\begin{minipage}{0.5\linewidth}
    \begin{tabular}{
            >{\columncolor[HTML]{FFFFFF}}c 
            >{\columncolor[HTML]{FFFFFF}}c 
            >{\columncolor[HTML]{FFFFFF}}c }
        题号 & 变量名称 & 变量类型 \\
        27 & 性别   & 分类   \\
        28 & 年龄   & 数值   \\
        29 & 学历   & 分类   \\
        30 & 年收入  & 分类  
    \end{tabular}

\end{minipage}
\subsubsection{统计学方法}
使用R语言作为工具,计数资料采用例数、百分比进行描述;计量资料采用均数、标准差进行描述,分类变量采用 $\tau$ 检验和单因素方差分析。

\section{结果}

\subsection{人口学特征}
265例中男性85人,女性180人。年龄区间<20岁的32人,20-30岁82人。30-40岁48人,40-50岁73人,50岁以上28人。
教育水平,小学学历2人,初中学历19人,高中学历50人,大学本科学历161人,研究生及以上学历33人。
年收入水平5万以下143人,5-10万72人,10-30万43人,30-50万4人,50万以上3人。

\subsection{得分特征}
采用均值-标准差分析,得出得分区间和得分率。

%知识维度
第3-10题,以各列均值替代缺失值,得出得分区间为$\hat{x}\pm s = 14.47 \pm 3.88$ ,平均得分率为 $0.761$。

%信念维度
第12-19题,以各列均值替代缺失值,得出得分区间为$\hat{x}\pm s = 12.10 \pm 3.48$ ,平均得分率为 $0.756$。

%信念维度
第12-19题,以各列均值替代缺失值,得出得分区间为$\hat{x}\pm s = 9.56 \pm 2.49$ ,平均得分率为 $0.398$。

分组检验的结果如下:

\begin{table}[]
    \centering
    \begin{tabular}{ccccc}
        \cellcolor[HTML]{FFFFFF}项目 & \cellcolor[HTML]{FFFFFF}人数  & \cellcolor[HTML]{FFFFFF}知识         & 信念         & 行为         \\
        \cellcolor[HTML]{FFFFFF}性别 & \cellcolor[HTML]{FFFFFF}    & \cellcolor[HTML]{FFFFFF}           &            &            \\
        \cellcolor[HTML]{FFFFFF}男  & \cellcolor[HTML]{FFFFFF}85  & \cellcolor[HTML]{FFFFFF}14.02±3.77 & 11.69±3.45 & 8.78±2.30  \\
        \cellcolor[HTML]{FFFFFF}女  & \cellcolor[HTML]{FFFFFF}180 & \cellcolor[HTML]{FFFFFF}14.68±3.93 & 12.28±3.49 & 9.93±2.59  \\
        \cellcolor[HTML]{FFFFFF}t值 & \cellcolor[HTML]{FFFFFF}    & \cellcolor[HTML]{FFFFFF}2.319      & 1.234      & 2.291      \\
        P值                         &                             & 0.021                              & 0.218      & 0.023      \\
        &                             &                                    &            &            \\
        教育水平                       &                             &                                    &            &            \\
        小学                         & 2                           & 8.89±2.49                          & 2.00±3.96  & 3.50±0.99  \\
        初中                         & 19                          & 14.21±3.82                         & 13.01±3.65 & 9.04±2.38  \\
        高中                         & 50                          & 14.32±3.85                         & 11.54±3.44 & 9.99±2.61  \\
        本科                         & 161                         & 14.64±3.92                         & 12.23±3.46 & 9.47±2.47  \\
        研究生及以上                     & 33                          & 14.35±3.85                         & 12.39±3.47 & 10.00±2.60 \\
        F值                         &                             & 3.867                              & 4.767      & 1.639      \\
        P值                         &                             & 0.005                              & 0.001      & 0.165      \\
        &                             &                                    &            &            \\
        年龄组                        &                             &                                    &            &            \\
        \textless{}20              & 32                          & 14.07±3.77                         & 11.36±3.31 & 8.16±2.15  \\
        20-30                      & 82                          & 14.29±3.83                         & 12.16±3.43 & 8.77±2.30  \\
        30-40                      & 48                          & 14.20±3.82                         & 10.58±3.31 & 9.16±2.39  \\
        40-50                      & 73                          & 14.88±3.99                         & 13.07±3.67 & 10.90±2.83 \\
        50以上                       & 28                          & 14.95±4.00                         & 12.91±3.64 & 10.53±2.74 \\
        F值                         &                             & 1.609                              & 4.214      & 5.057      \\
        P值                         &                             & 0.173                              & 0.003      & 0.001      \\
        &                             &                                    &            &            \\
        收入水平(万元)                   &                             &                                    &            &            \\
        \textless{}5               & 143                         & 14.44±3.87                         & 12.08±3.45 & 9.22±2.41  \\
        5--10                      & 72                          & 14.80±3.97                         & 12.86±3.62 & 10.38±2.69 \\
        10--30                     & 43                          & 14.51±3.89                         & 12.02±3.63 & 10.04±2.61 \\
        30--50                     & 4                           & 12.50±3.39                         & 6.25±3.50  & 5.48±1.50  \\
        50以上                       & 3                           & 10.18±2.81                         & 3.32±3.20  & 4.52±1.23  \\
        F值                         &                             & 4.398                              & 8.547      & 3.863      \\
        P值                         &                             & 0.002                              & 0.000      & 0.005     
    \end{tabular}
\caption{分组检验结果}
\end{table}

\section{讨论}
\begin{enumerate}
    \item 近半数受访者表示没听说“治未病”一词。
    
    尽管学界关于“治未病”理论的探讨已经有近六十年的历史,但是根据这一样本,这一概念的普及率仍只有46\%。而对于听说过这一概念的受访者,对于“治未病”具体含义的理解,有64\%选择了“预防未发生疾病”、15\%选择了“生病后防止病情进展”、22\%选择了“病后预防疾病再次复发”。
    	
	将听说过“知信行”概念和没有听说过“知信行”分为两组,$H_0$假设为二者知信行得分无差异,t检验后发现P>0.05,无显著性差异,再通过二者的均值可以看出,知识维度高2.738分,信念维度高0.997,行为维度高2.683。可见听说过“治未病”一词的群体对知信行理论更加了解,态度更积极,行为参与度高。
	
    \item 养生信息获取渠道分析
    
    第11题询问了受调查者获取养生知识的途径。其中79人选择社区宣传、155人选择亲戚朋友推荐、145人选择医生、155人选择电视、176人选择互联网、24人选择其他途径。可以看出,社区宣传在其中占比较小,仅占约30\%。
	建议社区开展主动式的养生知识普及,向有实际需求但是获知信息能力弱的老年人介绍节气养生,食疗、养生按摩等基础知识。

	
	\item 知识、信念得分和行为得分的反差对比
	
	通过Pearson相关系数分析得到系数矩阵,知识-信念、知识-行为、信念-行为之间的相关系数分别为0.539,0.304,0.208,P均小于0.001,说明三者之间为正相关。
	得分率上,知识维度和维度得分率都较为满意,百分比分别为76.1和75.6,但是行为维度陡降至39.8,可见“信而不行”的情况。分析可能的原因有以下几点。
	一是受调查者身体健康,没有就医或者保健的行动欲望,因而尽管对治未病知识有一定了解,并且对中医治未病理念有信心,但是不会做出实际行动。二是在问卷前期访谈中,部分受调查者认为平常获取的养生信息可信度不高,半信半疑之间,不会付诸实际行动。三是由于部分受调查者意志力不高,无法执行接收到的养生信息。
	
\end{enumerate}
\printbibliography
\end{document}