\section{知信行理论模式简介及其应用}
人们已经知道,在知识、信念和行为之间存在着一条沟壑:例如吸烟者知道吸烟存在的健康风险,即使他们了解后秉持了这一信念,他们却很难戒除吸烟的行为。在治未病理论的传播过程中,我们希望受众不仅仅是习得普遍的客体的知识,还能对此形成内化的信念并且真正地把这些知识应用于指导自身的生活。

在众多的理论模型中,知信行理论模式是应用最为广泛的模型,阐释了人的知识增长可以促进信念的加深进而转变为人的行为。知信行理论模式是认知理论和动机理论等在教育中的应用,主要探究了知识、信念和行为三者之前的联系。其中,知识是基础,信念(态度)是动力,行为则是目标。\cite{黄敬亨2006健康教育学}

“知”是某些明确的、得到普遍承认的客体知识。在知信行模式中,”知“代表通过某些媒介来传播知识,从而让人们“认知”和分析对象。“信”是在认识的基础上树立健康的“态度和信念”。“行”是在健康意志的指导下转变行为或养成行为习惯。\cite{金新政2003}

基于此模式进行的对照组研究在护理医学、预防医学领域形成了一套相对完备的体系,例如应用于糖尿病、高血压、呼吸系统慢性疾病的患者院外自理,术后康复,常见疾病预防等等。通过知信行模式,患者的院外自理能力加强,生活质量得到了提高,有的患者通过这一教育模式还形成了新的适应自身特点的生活模式。但是,知信行模式在治未病知识传播和实践当中的应用尚且缺乏,这也为我们的研究提供了创新的前景和空间。