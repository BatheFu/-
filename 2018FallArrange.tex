\documentclass{article}
\usepackage{ctex}
\usepackage{url}
\usepackage{setspace}
\usepackage[left=2.5cm,right=2.5cm]{geometry}
\title{2018挑战杯工作计划以及相关安排}
\author{付裕刚}
\date{\today}
\begin{document}
    \maketitle
    \zihao{-4}
    \section{整体计划简述}
    我们的项目名称暂定为“治未病知信行调查以其互联网+路径探索”。如题,关键词一个是知信行,一个就是互联网+。
    
    理论层面,如同主文件(main.pdf)所写的,我们已经写了的是治未病的理论概述,知信行理论介绍,社区健康体系(已经不再需要),以及问卷设计。还需要补充的是治未病理论部分,增添和预防医学的区别,增添治未病理论和实际服务的桥梁:具体落实的项目,以及治未病的需求分析。另外还需要补充微信健康类平台特点、文章特点、传播策略特点等等,同样是做文献的梳理和综述。
    
    实际操作层面,需要完成的任务有两个,一是对于知信行现状进行问卷调查,并且得出一份统计描述报告,二是对于粉丝进行知信行网络干预,并且在干预的三个月后进行效果评估。
    
    需要解释的问题是,为什么我们要费这么大劲做平台,写文章,然后向其他平台,例如一临另一个大创团队“知国医”进行投稿呢?这是和我们研究希望涉及的人群有关的。我们的预期读者是对于治未病(狭义上讲即预防保健)有兴趣的普罗大众,次一级的读者是对于中医知识有一定兴趣的大众。他们的相同点是都属于中青年群体,有治未病产品或者服务的一定消费能力和欲望,并且能够使用微信,定期查看某些公众号推送内容。但是读者群体也会有一定差异,其中主要预期读者的文化程度较次一级预期读者水平较低一些,他们的核心关注点在于“养生内容”上。次一级预期读者是主要预期读者的子集,他们不仅关注养生,还关注其他的中医内容。因此创作的主要内容是养生热点,但是每篇文章同样提供外部链接以介绍其他中医知识。
    \section{统一编写方式}
    \begin{spacing}{0.7}
    本项目可能涉及的软件
    \begin{itemize}
        \item *Git/Github
        \item *Notepad++
        \item TexStudio
        \item Endnote/NoteExpress
    \end{itemize}
**其中星号为必须品
\newline
\end{spacing}
本项目涉及不少文字编写工作,需要统一的协作平台来提高效率,统一版本。为此使用Git开源版本控制工具以及Github线上平台完成协作任务。对于文字稿和\LaTeX 论文稿的编写统一使用Notepad++软件,以避免Microsoft记事本存在的Unicode编码干扰问题。

具体方法见统一库\url{https://github.com/BatheFu/backupclcp}当中名为,统一提交标准的Word文档。
    
    另外,对于负责文献梳理和综述写作的同学,可以使用Endnote或者NoteExpress文献管理工具来梳理已经阅读过的文献,并且管理做过的标注。对于读过的文献,修改统一库中的refer.bib文件并且上传提交。其中Bibtex的格式可以在百度学术获得,也可以在上述提到的《统一提交标准》当中看到图文说明。
    \section{人员安排}
    文献梳理和综述书写:付裕刚、缪佳、葛任洁
    
    微信推文书写:徐天泓、缪佳、胡淳淳、周宇、付裕刚
    
    插图不定期绘制:仇苏妍、*王亚文
    
    统计报告:*汤珂安、付裕刚
    
    \section{九月计划安排}
    \begin{itemize}
        \item 完成文献综述部分的补充,建立起完整的基本架构,梳理研究的脉络。
        \item 使用搜狗微信搜索平台跟踪近三个月的养生话题热点,并且根据此写主推送稿20篇。
        \item 根据教材、名家论文集、医案、医话等完成基本概念的“词条科普式”文章,作为治未病养生推送的增补材料40篇。
        \item 每周工作时间4-5小时,利用好水课/课间时间会变得轻松。
    \end{itemize}

谢谢各位对我的理解和支持!


\end{document}