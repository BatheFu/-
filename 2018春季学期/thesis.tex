\documentclass{article}
\usepackage{ctex}
\usepackage{graphicx}
\usepackage{longtable}
\usepackage{multicol}
\usepackage{multirow}
\usepackage[left=3cm,right=3cm]{geometry}
\graphicspath{{images/}}
\usepackage[backend=biber,style=gb7714-2015]{biblatex}
\addbibresource[location=local]{refer.bib}	
\title{南京市社区治未病“知信行”现状调查}
\author{付裕刚}
\date{\today}
\begin{document}
    \maketitle
\section{对象和方法}
\subsection{对象}
选取南京市五个社区,分属不同的五个区划,每个社区的人数在6-8千人。统计各社区楼栋数,根据居委会的提示,排除青年人或老年人占绝大多数的楼层,随机选取居民进行自填式问卷调查。共调查192户,回收问卷265份。其中男性85人,女性180人,比例为32:68。
\subsection{方法}
\subsubsection{问卷设计}
根据先前进行的访谈,以及可进行参考的文献\cite{cjw_1_2009}\cite{wcy2011}\cite{kam2005}并咨询专家自行设计问卷,包括治未病知识、信念和行为三个维度。

其中,知识维度包括10条单选和2条多选。其中第1题为筛选题,不计分。
第2题承接第1题的引导,为多选,考察治未病的含义,每个选项计1分,共三分。
第3-10题采用4级评分法(正确=2分,部分正确=1分,错误=0分,不清楚=NA)。共计19分。
第11题为多选题,了解了知识获取的途径,不计分。故知识维度共19分。

信念维度共8条单选,序号为12-19,使用4级评分法(相信/愿意/喜欢=2分,部分相信/部分愿意/部分喜欢=1分,不相信/不愿意/不喜欢=-2分、不好说=NA)。共计16分。

行为维度共6题。序号为20-26,其中第20题设置跳题逻辑,选择(A=0次)跳过21题,分值为0、1、2、3。21为多选,考察治疗手段。22-26为单选,同样采用差额计分,最高项2分。其中26有4小题。共计24分。

具体可见下表
\newpage

\begin{longtable}[]{@{}llll@{}}

题号 & 维度 & 题目类型 & 是否有跳题\tabularnewline
\hline
\endhead
1 & 知识 & 单选 & 是\tabularnewline
2 & 知识 & 多选 & 否\tabularnewline
3-10 & 知识 & 单选 & 否\tabularnewline
11 & 知识 & 多选 & 否\tabularnewline
12-19 & 信念 & 单选 & 否\tabularnewline
20 & 行为 & 单选 & 是\tabularnewline
21 & 行为 & 多选 & 否\tabularnewline
22-26 & 行为 & 单选 & 否\tabularnewline
\hline
\caption{表中各题维度、类型、跳题设置}
\end{longtable}

此外还有匿名个人信息统计条目:

\begin{longtable}[]{@{}lll@{}}

题号 & 变量名称 & 变量类型\tabularnewline
\hline
\endhead
27 & 性别 & 分类\tabularnewline
28 & 年龄 & 数值\tabularnewline
29 & 学历 & 分类\tabularnewline
30 & 年收入 & 分类\tabularnewline
\hline
\end{longtable}


\subsubsection{统计学方法}
使用R语言作为工具,计数资料采用例数、百分比进行描述;计量资料采用均数、标准差进行描述,分类变量采用 $\tau$ 检验和单因素方差分析。

\section{结果}

\section{人口学特征}
265例中男性85人,女性180人。年龄区间<20岁的32人,20-30岁82人。30-40岁48人,40-50岁73人,50岁以上28人。
教育水平,小学学历2人,初中学历19人,高中学历50人,大学本科学历161人,研究生及以上学历33人。
年收入水平5万以下143人,5-10万72人,10-30万43人,30-50万4人,50万以上3人。

\section{得分特征}
采用均值-标准差分析,得出得分区间和得分率。

%知识维度
第3-10题,以各列均值替代缺失值,得出得分区间为$\hat{x}\pm s = 14.47 \pm 3.88$ ,平均得分率为 $0.761$。

%信念维度
第12-19题,以各列均值替代缺失值,得出得分区间为$\hat{x}\pm s = 12.10 \pm 3.48$ ,平均得分率为 $0.756$。

%信念维度
第12-19题,以各列均值替代缺失值,得出得分区间为$\hat{x}\pm s = 9.56 \pm 2.49$ ,平均得分率为 $0.398$。

分组检验的结果见excel表。

\section{讨论}
\begin{enumerate}
    \item 近半数受访者表示没听说“治未病”一词。
    
    尽管学界关于“治未病”理论的探讨已经有近六十年的历史,但是根据这一样本,这一概念的普及率仍只有46\%。而对于听说过这一概念的受访者,对于“治未病”具体含义的理解,有64\%选择了“预防未发生疾病”、15\%选择了“生病后防止病情进展”、22\%选择了“病后预防疾病再次复发”。、
	
	将听说过“知信行”概念和没有听说过“知信行”分为两组,$H_0$假设为二者知信行得分无差异,t检验后发现P>0.05,无显著性差异,再通过二者的均值可以看出,知识维度高2.738分,信念维度高0.997,行为维度高2.683。可见听说过“治未病”一词的群体对知信行理论更加了解,态度更积极,行为参与度高。
	
    \item 养生信息获取渠道分析
    第11题询问了受调查者获取养生知识的途径。其中79人选择社区宣传、155人选择亲戚朋友推荐、145人选择医生、155人选择电视、176人选择互联网、24人选择其他途径。可以看出,社区宣传在其中占比较小,仅占约30\%。
	建议社区开展主动式的养生知识普及,向有实际需求但是获知信息能力弱的老年人介绍节气养生,食疗、养生按摩等基础知识。

	
	\item 知识、信念得分和行为得分的反差对比
	
	通过Pearson相关系数分析得到系数矩阵,知识-信念、知识-行为、信念-行为之间的相关系数分别为0.539,0.304,0.208,P均小于0.001,说明三者之间为正相关。
	得分率上,知识维度和维度得分率都较为满意,百分比分别为76.1和75.6,但是行为维度陡降至39.8,可见“信而不行”的情况。分析可能的原因有以下几点。
	一是受调查者身体健康,没有就医或者保健的行动欲望,因而尽管对治未病知识有一定了解,并且对中医治未病理念有信心,但是不会做出实际行动。二是在问卷前期访谈中,部分受调查者认为平常获取的养生信息可信度不高,半信半疑之间,不会付诸实际行动。
	
\end{enumerate}


\printbibliography
\appendix{}
 \section{治未病问卷小样}
您好,这是一份由南京中医药大学第一临床医学院发起的调查问卷,旨在了解社区居民对于中医治未病理论的认知、行为等情况,响应“健康中国”工程,为后面中医治未病理论的传播做一个描述性统计分析。

本问卷时长约五分钟,我们郑重承诺,本问卷的所有数据仅用于科研用途,您的所有个人信息将受到相关法律的保护。

\begin{enumerate}
%知识部分
\item 中医学特点之一是整体观,认为“天人合一”,即人的健康受到周围环境(自然、社会、家庭)的影响。您认为:

A.正确\qquad B.部分正确\qquad C.错误 \qquad D.不清楚

\item 中医学认为人应该调节身体状况以适应气候、环境的变化。您认为:

A.正确\qquad B.部分正确\qquad C.错误 \qquad D.不清楚

\item 人的身体是一个整体,一个系统或器官的疾病可以影响其它系统或器官。您认为:

A.正确\qquad B.部分正确\qquad C.错误 \qquad D.不清楚

\item 人的情绪、精神状态对疾病有影响。您认为:

A.正确\qquad B.部分正确\qquad C.错误 \qquad D.不清楚

\item 人的健康与饮食有很大关系,中医的食疗指通过饮食调理来改变人的健康状况。
您认为:

A.正确\qquad B.部分正确\qquad C.错误 \qquad D.不清楚

\item 食物有寒凉等不同属性,人的体质也有阴阳等差别,所以饮食要根据体质不同进行选择。这个观点您认为:

A.正确\qquad B.部分正确\qquad C.错误 \qquad D.不清楚

\item 治未病理论认为“春夏养阳,秋冬养阴。”,因此夏天不要怕晒太阳,冬天不要大汗淋漓,这个观点您认为:

A.正确\qquad B.部分正确\qquad C.错误 \qquad D.不清楚

\item 刮痧、拔火罐、推拿等方法是治未病理论的运用,可以起到增强体质、保护健康的作用。这个说法您认为:

A.正确\qquad B.部分正确\qquad C.错误 \qquad D.不清楚

\item 各人的体质不同,所以他们应该根据自身的体质选择不同的饮食。这个说法您认为:

A.正确\qquad B.部分正确\qquad C.错误 \qquad D.不清楚

\item 有人说偏方也是是中医的一种,也能起到治病或保健的作用,您认为,

A.正确\qquad B.部分正确\qquad C.错误 \qquad D.不清楚

\item
草药煎汤服用是中医最常用的治疗方法,那么煎药时间越久越好,您认为,

A.正确\qquad B.部分正确\qquad C.错误 \qquad D.不清楚

\item 
治未病是中医特有的理论和方法,西医没有这样的说法,您认为,

A.正确\qquad B.部分正确\qquad C.错误 \qquad D.不清楚

\item 中医的存亡取决于中医的治疗是否有效,您认为,

A.正确\qquad B.部分正确\qquad C.错误 \qquad D.不清楚

%信念部分
\item 
您相信中医理论有其科学性,能从和西医不同的角度解释和解决疾病问题吗?

A.相信\qquad B.部分相信\qquad C.不相信\qquad D.不好说

\item 您本人能接受中医对您的诊断吗?

A.相信\qquad B.部分相信\qquad C.不相信\qquad D.不好说

\item 中医药的治疗(包括针灸、推拿、中药、中成药)只要选择得当,对治疗大多数疾病有效果,您相信吗?

A.相信\qquad B.部分相信\qquad C.不相信\qquad D.不好说

\item 您本人愿意接受中医药治疗方法吗?

A.愿意\qquad B.部分愿意\qquad C.不愿意 \qquad D.不好说

\item 您本人愿意尝试治未病提供的运动保健(太极拳、气功等)的方法吗?

A.愿意\qquad B.部分愿意\qquad C.不愿意 \qquad D.不好说

\item 您会用中医有关调畅情志的理论和方法来解决心理、情绪上的问题吗?

A.愿意\qquad B.部分愿意\qquad C.不愿意 \qquad D.不好说

\item 您愿意学习中医饮食调理的理论和方法并在日常饮食中加以运用吗?

A.愿意\qquad B.部分愿意\qquad C.不愿意 \qquad D.不好说

\item 
您对于当前国家出台政策,鼓励中医药发展怎么看?

A.我支持\qquad B.我保持中立\qquad C.我反对

\item 您喜欢看养生方面的书籍、视频等资料吗?

A.喜欢\qquad B.一般\qquad C.不喜欢\qquad D.我不清楚

%行为题目
\item
过去一年您接受中医治疗的次数?

A.0次 \qquad
B.1-2次\qquad
C.2-5次\qquad
D.5次以上

\item
您接受了哪一种中医治疗呢?

A.草药煎汤\qquad B.贴敷的膏药\qquad C.物理疗法,比如针灸,拔火罐等\qquad D.填写\underline{\makebox[6em]{}}

\item 您过去服用过几次膏方(固元膏、阿胶膏等等)吗?

A.0次\qquad B.1-2次\qquad C.3-5次\qquad D.5次以上

\item 您自己的孩子或者周围人的孩子接受过“三伏贴”来治疗肺部疾患吗?

A.有\qquad B.没有\qquad C.不清楚

\item 
您近一年来参加过几次养生知识的讲座?

A.0次\qquad B.1-2次\qquad C.3-5次\qquad D.5次以上

\item 
未来您会去接受中医治疗吗?

A.会\qquad B.不会\qquad C.可能

\item 
我们知道,电视台会播放中医节目,也有微信文章进行中医知识的传播。
\subitem 
   您收听/观看此类节目吗?
   
    A.从不\qquad B.偶尔\qquad C.经常

	\subitem 
    您阅读此类文章吗?
	
    A.从不\qquad B.偶尔\qquad C.经常
    
    \subitem 
    您会仿效节目/文章中的做法吗?
	
	A.从不\qquad B.偶尔\qquad C.经常
    
    \subitem 
   您能坚持节目/文章里面的做法吗?
   
    A.我能坚持做\qquad B.坚持不久就放弃了\qquad C.我没做过
    
\item 
您的
性别\underline{\makebox[6em]{}}
年龄\underline{\makebox[6em]{}}
学历\underline{\makebox[6em]{}}
年收入情况\underline{\makebox[6em]{}}
\newline
感谢您的参与以及对我们的支持!
\end{enumerate}

\end{document}