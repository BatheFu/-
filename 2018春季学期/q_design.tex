\section{治未病“知信行”调查问卷的设计}
该问卷主要评价受调查者对中医治未病理论对于中医治未病理论的客观认知、所持信念以及行动意愿,以得出描述性的分析。设计的条目先进行征集,经编写后交付相关专家进行评议,并进行问卷重测以检验其信度。
\subsection{先前参考}
先前的可进行参考的文献有:
陈建伟对社区老年人进行的调查\cite{cjw_1_2009},论文中包含了一份针对老人的“知信行”调查问卷,选取部分条目进行参考。;吴朝阳\cite{wcy2011}编制的《中医态度量表》,该研究对北京中医药大学的200名学生进行了问卷研究;
Kam-Lun Ellis Hon\cite{kam2005}编制了态度问卷并针对780名香港医学院学生进行的态度问卷调查。
\subsection{设计原则}
问卷整体的架构依照知识、信念(态度)、行为三个层次展开,每个部分的题量在10题左右。随机变量设置为定序或定性。

知识部分主要测验被试者对于中医背景知识和治未病相关知识的了解程度,并对我们提供的信息进行判断。

信念部分题目的设置出于两方面的考虑。第一种是对于中医/治未病理论及治疗的相信度;另一种是对于参与中医/治未病活动的愿意程度。

行为部分主要用于了解被试者在过去的一段时间内是否有参与到治未病相关的活动当中去。这些活动包括获知治未病理论或者实践的知识信息,如何了解到治未病有关的活动,以及实际参与治未病活动。考虑到有相当一部分人是在生病时接受治疗的时候获知了一些治未病养生保健的知识,因此我们设置的题干会和中医有很大的联系。


