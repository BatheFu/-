\documentclass{ctexart}
\date{February 2018}
\usepackage{graphicx}
\usepackage{setspace}
\usepackage{geometry}
\bibliographystyle{unsrt}
\graphicspath{{C:/Users/Dell/Documents/}}
\title{网络舆论对大众对中医态度的引导作用——以南京市为例}
\author{{付裕刚\thanks{fuyugang11@foxmail.com}}}

\begin{document}

\maketitle
\pagenumbering{roman}
\begin{abstract}
	目前,微信,微博,论坛,贴吧等自媒体成为传播中医知识和文化的重要载体。我们可以从自媒体中获得客观知识,情绪表达,行为观点等等。自媒体改变了以往只能被动接受信息的新闻模式,而让更多的人参与生成带有个人印记的作品。其中,微博是群众参与广泛,最有创造性的发声平台之一。
	
	本文采用数据采集与分析的方法对微博上人群日常生活之中与中医相关的内容进行采集,并且在这些内容中以关键词的方法分析出人群对中医的实际态度,并进行实地问卷调查获知现实选择,并通过一定的换算得出影响因子。并由此为中医的互联网传播给出来自内容长和政策上的建议。
	
	由于使用微博数据采集,本文具有更好的“盲采”特征,能够一定程度避免实地调查带来的倾向性,但是不可避免的获得样本没有现实情境复杂,譬如高年龄段的人群网络发言频次较低等等。
\end{abstract}
\tableofcontents
\newpage
\pagenumbering{arabic}

\section{对题目的思考}
\subsection{研究背景}
改革开放以来,人民大众的生活水平逐步提高,精神世界日益丰富。信息时代的到来使得大众获取信息的渠道变得更加便捷、广阔。同时,互联网催生了公共平台的诞生,大型的社交平台事实上成为公民讨论时事,分享新知的场所。可以说,例如微博一样的大型社交平台是广泛舆论的聚集地,可以反映出民众对于特定事物的认知、理解、认同。

但与此同时,互联网的舆论焦点随着热度转移,碎片化阅读成为人们追捧的对象,海量信息蜂拥而至,真实知识踪迹难寻。陈奕指出,碎片化阅读具有传播麻醉作用,观众作为传播的受众只关注“是什么”,继而跳跃至下一个信息,而抛弃了任何逻辑思考的可能。我们可以看到,该环境传播者的地位居高,某些经过精心编写的内容在经过适当的推动可以发挥巨大的传播效应;而受众的地位居低,经过麻醉传播效应\cite{cy2014}的放大后,增大了盲信的概率,更易助长错误信息的扎根。

在这样的环境中,传播者为了达到“吸睛”的目的,对内容的遴选不加审视。传统中医因其理论的复杂性和它所在的语言体系和公共话语体系的差别令普通人对此望而却步,自然易被其他中医文化的衍生物替代。

何为中医文化的衍生物呢?在这里,我们澄清传统中医和中医文化的概念。前者是相对狭义的概念,单指在中医理论指导下的医术,具体形式有处方,针灸,拔罐等。而后者是广义的概念,指的是原始宗教、巫术、奇闻异事、民间传说等或催生出的中医文化。当前阶段的中医文化又是由上一阶段中医文化衍生出的文化产品。

\subsection{研究思路}
\includegraphics[width=\textwidth]{tzb2.jpg}

\subsection{研究意义}
不可否认的是,当前网络上传播的中医信息大多存在夸大或贬低等扭曲的特点,一定程度上误导了大众对中医的正确认识。评估网络舆论对人们对中医态度的影响能够对如何正确开展中医知识传播起指导作用。另外,本文从实际出发,采用实地问卷调查,对网络态度结果和现实态度进行比较,从而换算出一个网络舆论对于现实的影响因子。


\subsection{网络舆论}
舆论的产生可能是群体事件、有影响力的作品、言论等。由此,搜集舆论就是对关键事件和广泛言论的采集,重点关注引发热议群体事件。

对于重点事件的搜集较为容易,只需要查询近三年和中医相关的新闻事件即可。但是我很怀疑新闻对于中药的讨论热情远远胜于中医本身。

对于广泛言论的搜集,取材从微博较能反映更多层次群体的认知态度,统计词频和言论相关性,比较地区差异,再结合实地问卷的调查得出结论。在进行数据筛选的时候,具体的内容上要把中医和中药分开。然后,研究这些态度的地域特点来对中医正确认知做出一些建议。

\subsection{中医态度}
对于这个名词的理解我们很有可能和民众发生差异。我们所理解的中医是单论治疗的可行性,包括利用草药组方,尤其强调药物的配伍以及物理疗法(针灸、拔罐、推拿等)。而民众可能对某株特殊的神药更感兴趣或是对一个超脱中医理论的大方剂颇为追捧,其实质还是对多就是好的盲信。
	
由于网络上中医和中药常被结合起来讨论,那么中医一词的范畴在我们这里和群体认知那里可能会有差异。因而在设计问卷时面临是否明确概念的问题。

\subsection{前人的研究}
在中国知网上以“中医态度”为关键词搜素仅得论文一篇《中医态度量表的编制》\cite{wcy2011}。

该论文以北京中医药大学校内学生为对象进行了初步的定量分析,尽管他的问卷设计较简陋,但对我们的问卷设计扔有一定的参考价值。

这篇论文的研究范围是中医药院校的学生,样本群体也不大,这样的人群本身具有特殊性,对中医具有天生的亲和,由此研究结果不能作为对普遍人群适用的态度量表。

另外,药效是中医药的生命,有效的治疗显然会表现出正向的态度。由此再由“中医的需求性“为关键词进行搜索,所获较多。大多采取问卷或访谈方法。如沈燕萍\cite{syp2006}等对杭州周边农村进行的调查显示出人们对中西药物的优缺点的认识等,但尚未反映出整体的抽离的态度,而是混杂在这些对医药的认识之中。

\subsection{研究方法}
社会学的基本方法有定性和定量两种。目前,中医态度方面的研究,定量先例少并且难度较高,定性研究倘若时间仓促,则又容易流于空泛。挑战杯时间相对充裕,因此定性研究较为合适,因此我才想使用问卷的方法,舍弃了量表的制定。这样,我们需要对可能出现的大量态度、观点进行梳理,然后观察这些观点出现的频率。网络采集数据需要关键词,这些词语的选取该如何进行还需要进一步的考虑。

\subsection{不足和展望}
资源所限,实地调查的范围聚集在南京市的某些点,如若将来能扩大范围,则测出的影响因子更加准确。

另外,中医是传统文化的一部分,又历经千年的实践,具有同农业社会一样的超稳定特征。我们尤其期待千禧一代对于中医的态度如何。若有再深入研究的机会,探索传统,尤其是有语境隔阂的情况下,如何在新媒体新时代更好发展。

\section{论文的整体框架}
    \subsection{近三年中医热点事件观点回顾}
    \subsection{近三年微博上中医相关言论词频统计和分析}
\begin{itemize}
	\item 选词的依据
	\item 当前语境和中医语境的差异
	\item 数据爬取和结果分析 
\end{itemize}

\section{态度问卷的设计}
\subsection{设计原则}
总的来说,设计问卷需要
明确每组子问题的目的,并且使用统计方法检验问卷信度\\

下面的几点,主要截取改编自布拉德伯恩的《态度问卷手册》\S1.4。\cite{bld2011}\\

\begin{enumerate}
\item 确定要测量的态度已被清楚的说明
\item 确定测量的态度的关键方面的主要成分,如认知、评价和行为方面的主要成分,不要假设他们是一致的。
\item 把分离的问题和和一系列衔接的问题结合起来
\item 行为意图可以直接询问,也可以做可能性的测量
\item 避免概念的多重性和答案的非唯一性,除非是多项选择
\item 对于备选项的陈述将很大程度影响回答
\item 对新的态度问题进行前测,前测中使用折半测量法
\item 在问具体的态度问题前可以先问一般的问题,即普遍问题优先原则
\end{enumerate}
\newpage

\subsection{相关态度样卷参考}
我们从Hon等人的研究\cite{kam2005}选取了问题2,3,5。

\begin{quote}
	问题2:\\
	2) How often did you consult the Chinese medicine practitioner in the past 1 year?
	
	问题3:\\
	3) In what circumstance did you use TCM?\\
	A. Upper respiratory tract infection e.g. common cold, flu, sore throat, etc.\\
	B. Skin problem\\
	C. Gastrointestinal problem\\
	D. Others (please specify)
	
	问题5:\\
	5) What kind of TCM treatment do you use? (can circle more than 1 item)
	A. Herbal tea or soup\\
	B. Over-the-counter Chinese medicine\\
	C. Cream or ointment consisted of TCM\\
	D. External application of herbs\\
	E. Others (e.g. acupuncture) 
\end{quote}

前人使用过的问题较为安全可靠。但是遗憾的是,Hon的研究没有说明每个问题或者每组子问题的设计目的。

\section{问卷小样}
您好,这是一份由南京中医药大学第一临床医学院发起的调查问卷,旨在了解当前大众对于中医的认知和态度,并与互联网的调研结果进行对比,促进高校和各类媒体制定更符合实际的中医传播方案。

本问卷时长约五分钟,我们郑重承诺,本问卷的所有数据仅用于科研用途,您的所有个人信息将受到相关法律的保护。

\begin{enumerate}
\item
您认为中成药也可以划归到中药里面吗?\\
A.是\\
B.不是\\
C.我不确定

\item
您觉得偏方是中医的一种吗?\\
A.是\\
B.不是\\
C.我不确定

\item
您清楚煎药的方法吗?\\
A.是的\\
B.不是\\
C.我不确定

\item
您认为食疗是中医的治疗手段吗?\\
A、是\\
B、不是\\
C、我不确定

\item
过去的一年里,您接受过中医治疗吗?(Hon第一题)\\
A. 是的\\
B. 没有(请跳转至第9题)

\item
过去一年您接受中医治疗的次数?\\
A.1-2次\\
B.2-5次\\
C.5次以上

\item
您接受了哪一种中医治疗呢?\\
A.草药煎汤\\
B.贴敷的膏药\\
C.物理疗法,比如针灸,拔火罐等\\
D.填写

\item 
下面哪些情况您会使用中药呢?(改编Hon第三题)\\
A.普通感冒,嗓子疼,咳嗽等\\
B.皮肤疾病\\
C.胃肠疾病\\
D.(填写)\underline{\makebox[6em]{}}

\item 
未来您还会去接受中医治疗吗?\\
A.会\\
B.不会\\
C.可能

\item 
相较西医,您认为当下中医治疗费用高吗?\\
A.高\\
B.一般\\
C.低\\

\item 
印象中您觉得中医治疗疗效慢吗?\\
A.缓慢\\
B.一般\\
C.不慢\\

\item 
有人说“中医的存亡取决于是否有效”,您认同这一说法吗?
A.认同\\
B.不认同\\
C.我没听说过

\item 
有人说“中医的存亡取决于药材质量”,您认同这一说法吗?
A.认同\\
B.不认同\\
C.我没听说过

\item 您喜欢看养生方面的书籍、视频等资料吗?
A.喜欢\\
B.一般\\
C.不喜欢\\

\item 
您对于当前国家鼓励中医药发展怎么看?\\
A.我支持\\
B.我保持中立\\
C.我反对

\item 
有人建议用颗粒冲剂取代煎汤药,您支持这一行为吗?\\
A.我支持\\
B.我保持中立\\
C.我反对

\item 
我们知道,电视台会播放中医节目,也有微信文章进行中医知识的传播。
\begin{enumerate}
	\item 
   您收听/观看此类节目吗?\\
    A.从不\\
    B.偶尔\\
    C.经常

\item 
    您阅读此类文章吗?\\
    A.从不\\
    B.偶尔\\
    C.经常
    
    \item 
    您喜欢这些节目/文章吗?\\
    A.喜欢\\
    B.一般\\
    C.不喜欢
    
    \item 
    您会仿效节目/文章中的做法吗?\\
    A.从不\\
    B.偶尔\\
    C.经常
    
    \item 
   您能坚持节目/文章里面的做法吗?\\
    A.我能坚持做\\
    B.坚持不久就放弃了
	\end{enumerate}
\item 
您的
性别\underline{\makebox[6em]{}}
年龄\underline{\makebox[6em]{}}
学历\underline{\makebox[6em]{}}
年收入情况\underline{\makebox[6em]{}}
\newline
感谢您的参与以及对我们的支持!
\end{enumerate}


\subsection{实地问卷发放与回收}
\subsection{网民身份/实际身份的大众对中医态度的认知比较}

\begin{itemize}
	\item 横向对比
	\item 影响因子的计算和得出
	\item 对互联网中医传播的建议
\end{itemize}


\section{难点}

\begin{enumerate}
	\item 理论缺失:目前我们对于社会学发展脉络和研究理论不清晰;以及对
	人群的态度进行量化研究的社会学以及社会心理学理论不清晰。
	\item 问卷的信度:设计问卷和调查态度的量表有难度,主要是问题的选择和量表的改造。
	\item 问卷的设计: 目前已经设计了初步问卷。
	\item 问卷的信度:采用统计学方法。
	\item 数据爬取:微博有着严密的反爬虫机制,微博数据的爬取会是一个难点。
\end{enumerate}

\bibliography{tzb}
\end{document}

