\documentclass[12pt]{ctexart}
\CTEXsetup[number={\chinese{section}},format={\Large\bfseries}]{section}
\title{2018年暑期社会实践策划书}
\author{项目负责人:付裕刚\\南京中医药大学\quad 第一临床医学院}
\date{}
\begin{document}
    \maketitle
    \clearpage
    \tableofcontents
    \clearpage

    \vspace{-10 pt}
    \section{活动背景}
    健康管理是对个体或群体的健康进行全面监测、分析、评估、提供健康咨询和指导以及对健康危险因素进行干预的全过程。健康管理的宗旨就是充分调动个体和群体以及整个社会的积极性,有效利用有限的资源来达到最大的效果。
    
    1929年,美国蓝十字和蓝盾保险公司进行疾病管理实践与探索中首次提出健康管理理念。1969年,美国政府将健康维护组织纳入国家医疗保障体系中,并于1971年为其立法,逐步形成了一个系统化管理的健康观。上世纪末,健康管理的理念开始引进我国。
    
    在2008年卫生部全国卫生工作会议上正式提出实施“健康中国2020”战略,其目标是到2020年,建立覆盖城乡居民的比较完善的基本医疗卫生制度,缩小经济、社会发展水平差异造成的健康不平等现象,实现“人人享有基本医疗卫生服务目标,促进卫生服务利用的均等化,大幅提高全民健康水平”。
    
    党的十八届五中全会提出了“推进健康中国建设”的发展战略,特别突出以人的健康为中心。我国的医疗卫生事业的改革方向已从解决人民群众看病就医问题向促进和保障人民健康转变。
    
    此外,改革开放以来,随着经济条件的改善,人们开始注意到身心健康是生活安逸的重要保障,但是与此同时市场上养生保健方法鱼目混珠,一方面这些方法不够理论化、体系化,发挥作用寥寥;另一方面有的方法违背客观事实,有人甚至夸大作用,以此来谋求不当之财,污名化中医养生的应有作用。
    
    对于此类情景,我们应当正本清源,普及系统化科学的治未病理论,使它得以在新的时代发挥应有的活力,指导人们合适的生活调节方式,为新时期的预防保健工作做出系统性的贡献。
    \section{活动目的}
    本次调查以南京市的一部分社区为样本采集点,致力于对当前人们对于中医治未病理论的认知、信念以及行动意愿进行一系列的数据搜集,简称为“知信行”调查,旨在为后续的中医治未病理论的传播研究开展铺垫,最后形成一个完整的描述性的统计分析。
    \section{活动主题}
    走进基层:社区居民治未病意识探寻
    \section{活动时间、地点}
    活动时间:2018年7月13-28日
    
    活动地点:栖霞区、玄武区、建邺区、鼓楼区、秦淮区各一社区
    \section{活动分工及负责人}
    \subsection{团队介绍}
    \begin{table}[ht]
        \centering
        \begin{tabular}{|l|l|l|l|l|}
            \hline
          指导老师  & 姓名 & 职称 & 专业 & 单位 \\ \hline
            & 汤少梁 & 教授 & 电子商务 & 卫生经济管理学院 \\ \hline
           学生 & 姓名 & 院系 & 专业 & 联系方式 \\ \hline
            & 付裕刚 & 一临 & 16中医五 & 18252057760 \\ \hline
            & 葛任洁 & 一临 & 16中医五 & 17314990979  \\ \hline
            & 陈薇羽 & 卫管 & 16公管 & 18252058705
             \\ \hline
            & 胡濛 & 一临 & 16中医五 &18851156108  \\ \hline
            & 周宇 & 一临 & 16中医五 & 18252057957
             \\ \hline
        \end{tabular}
    \end{table}
    \subsection{具体分工}
    指导老师:汤少梁 
    
    总负责人:付裕刚
    
    活动策划:付裕刚、葛任洁
    
    问卷编制:付裕刚、葛任洁
    
    社区联络与沟通:葛任洁
    
    问卷发放和回收:付裕刚、葛任洁、陈薇羽、胡濛、周宇
    
    照片拍摄和剪辑:周宇
    
    文字稿整理:陈薇羽、付裕刚
    
    采访和录音:胡濛、付裕刚
    \section{活动形式}
    社区走访,问卷调查、访谈
    \section{活动内容}
    \subsection{活动前期准备}
    \begin{enumerate}
        \item 设计问卷:由中医专业的付裕刚、葛任洁同学完成初步的条目设计并且参照陈建伟、吴朝阳、Kam-Lun Ellis Hon等人对于中医社区调查以及医学生中医态度的研究选取或者改编部分条目,以贴合治未病的主题。然后提交给相关的专家老师进行筛选,结合老师给出的筛查建议来确定最终的条目。
        \item 确定社区:由葛任洁同学联系相关的社区并确定实际调研的日期。选取的社区具有基数较大,人群分布比例合理的特点。由于选取抽样调查的方式,我们一共走访南京市不同区域的5个社区,每个社区发放问卷100份,共计500份调查问卷。
        \item 打印样卷,选取一个小样本的人群进行测试,对于被反映有语义/选项模糊的条目进行修改或者删减。
        \item 打印最终的五百份调查问卷,确定好走访各个社区的具体日期和时间段。
        \item 人员培训:对于实际发放当中的措辞和具体操作进行规范,避免不愉快的矛盾的发生。   
    \end{enumerate}
   \subsection{活动主要安排}
   在实际调研阶段,团队主要的主要任务是发放问卷,在约定的时间后对问卷进行回收。之后经过一段时间的间隔,时间允许的情况下,可以做一次重测。如果时间紧还可以在开学后接着完成。为了追踪调查对象的情况,团队将要进行挨家挨户的拜访,并且对于接受问卷调查的部分人群,做一些围绕治未病政策/做法/理论为主题的访谈,以积累更多的调研资料。
   
   拟定的日程安排如下:
   
   7月13日:邀请指导老师开“走进基层:社区居民治未病意识探寻”暑期社会实践活动动员大会
   
  7月14-15日:准备好前期需要的材料,和社区商量妥当。
  
  7月16-17日:走访调研栖霞区社区
  
  7月18日:总结第一次走访的经验,调整发放问卷策略、询问策略、访谈策略。
  
  7月19-20日:走访调研玄武区社区
  
  7月21-22日:走访调研建邺区社区
  
  7月23日:总结第二次走访的经验,调整发放问卷策略、询问策略、访谈策略。
  
  7月24-25日:走访调研秦淮区社区
  
  7月26-27日:走访调研鼓楼区社区
  
  7月28日:召开工作总结大会,对此次暑期社会实践进行回顾分析,总结经验教训。
   
  \subsection{活动后期处理}
  \begin{enumerate}
      \item 对收集的问卷整理成电子形式,按照统一的数据格式,并做简单的数据分析。
      \item 整理各个环节拍摄的相关照片
      \item 做好相应的文字稿记录和录音材料的归档
  \end{enumerate}
  
\section{经费预算}

问卷打印费用:1*500 = 500元

给受调查者的小礼品 2*150 = 300元

交通费用: 200元


\end{document}