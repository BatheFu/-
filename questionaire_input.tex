\section{问卷小样}
您好,这是一份由南京中医药大学第一临床医学院发起的调查问卷,旨在了解当前大众对于中医的认知和态度,并与互联网的调研结果进行对比,促进高校和各类媒体制定更符合实际的中医传播方案。

本问卷时长约五分钟,我们郑重承诺,本问卷的所有数据仅用于科研用途,您的所有个人信息将受到相关法律的保护。

\begin{enumerate}
\item
您认为中成药也可以划归到中药里面吗?\\
A.是\\
B.不是\\
C.我不确定

\item
您觉得偏方是中医的一种吗?\\
A.是\\
B.不是\\
C.我不确定

\item
您清楚煎药的方法吗?\\
A.是的\\
B.不是\\
C.我不确定

\item
您认为食疗是中医的治疗手段吗?\\
A、是\\
B、不是\\
C、我不确定

\item
过去的一年里,您接受过中医治疗吗?(Hon第一题)\\
A. 是的\\
B. 没有(请跳转至第9题)

\item
过去一年您接受中医治疗的次数?\\
A.1-2次\\
B.2-5次\\
C.5次以上

\item
您接受了哪一种中医治疗呢?\\
A.草药煎汤\\
B.贴敷的膏药\\
C.物理疗法,比如针灸,拔火罐等\\
D.填写

\item 
下面哪些情况您会使用中药呢?(改编Hon第三题)\\
A.普通感冒,嗓子疼,咳嗽等\\
B.皮肤疾病\\
C.胃肠疾病\\
D.(填写)\underline{\makebox[6em]{}}

\item 
未来您还会去接受中医治疗吗?\\
A.会\\
B.不会\\
C.可能

\item 
相较西医,您认为当下中医治疗费用高吗?\\
A.高\\
B.一般\\
C.低\\

\item 
印象中您觉得中医治疗疗效慢吗?\\
A.缓慢\\
B.一般\\
C.不慢\\

\item 
有人说“中医的存亡取决于是否有效”,您认同这一说法吗?
A.认同\\
B.不认同\\
C.我没听说过

\item 
有人说“中医的存亡取决于药材质量”,您认同这一说法吗?
A.认同\\
B.不认同\\
C.我没听说过

\item 您喜欢看养生方面的书籍、视频等资料吗?
A.喜欢\\
B.一般\\
C.不喜欢\\

\item 
您对于当前国家鼓励中医药发展怎么看?\\
A.我支持\\
B.我保持中立\\
C.我反对

\item 
有人建议用颗粒冲剂取代煎汤药,您支持这一行为吗?\\
A.我支持\\
B.我保持中立\\
C.我反对

\item 
我们知道,电视台会播放中医节目,也有微信文章进行中医知识的传播。
\begin{enumerate}
	\item 
   您收听/观看此类节目吗?\\
    A.从不\\
    B.偶尔\\
    C.经常

\item 
    您阅读此类文章吗?\\
    A.从不\\
    B.偶尔\\
    C.经常
    
    \item 
    您喜欢这些节目/文章吗?\\
    A.喜欢\\
    B.一般\\
    C.不喜欢
    
    \item 
    您会仿效节目/文章中的做法吗?\\
    A.从不\\
    B.偶尔\\
    C.经常
    
    \item 
   您能坚持节目/文章里面的做法吗?\\
    A.我能坚持做\\
    B.坚持不久就放弃了
	\end{enumerate}
\item 
您的
性别\underline{\makebox[6em]{}}
年龄\underline{\makebox[6em]{}}
学历\underline{\makebox[6em]{}}
年收入情况\underline{\makebox[6em]{}}
\newline
感谢您的参与以及对我们的支持!
\end{enumerate}
